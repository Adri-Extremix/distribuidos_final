\documentclass[]{article}
\usepackage{graphicx}
\usepackage[spanish]{babel}
\usepackage[a4paper, top=2.5cm, bottom=2.5cm, left=3cm, right=3cm]{geometry}
\usepackage[hidelinks]{hyperref}
\usepackage{listings}
\usepackage{xcolor}


\lstset{
  language=C,
  basicstyle=\ttfamily\small,
  commentstyle=\color{gray},
  keywordstyle=\color{blue},
  stringstyle=\color{green},
  numberstyle=\tiny\color{gray},
  stepnumber=1,
  showspaces=false,
  showstringspaces=false,
  tabsize=2,
  breaklines=true, captionpos=false,
}

%title
\title{Práctica 1} 

\author{Adrián Ferández Galán y César López Mantecón}

\begin{document}

\begin{titlepage}
    \centering
   \includegraphics[width=0.9\textwidth]{uc3m.jpg} 
    {\Huge Universidad Carlos III\\
    
     \Large Sistemas Distribuidos\\
     \vspace{0.5cm}
     Curso 2023-24}
    \vspace{2cm}

    {\Huge \textbf{Práctica Final} \par}
    \vspace{0.5cm}
    {\Large Diseño e implementación de un sistema peer-to-peer \par}
    \vspace{8cm}

   \textbf{Ingeniería Informática, Tercer curso}\\
    \vspace{0.2cm} 
    Adrián Fernández Galán (NIA: 100472182, e-mail: 100472182@alumnos.uc3m.es) \\
    César López Mantecón   (NIA: 100472092, e-mail: 100472092@alumnos.uc3m.es)
    \vspace{0.5cm}

   
    \textbf{Prof .} Félix García Caballeira y Alejandro Calderón Mateos\\
    
    \textbf{Grupo: } 81   
    
\end{titlepage}
\newpage

\renewcommand{\contentsname}{\centering Índice}
\tableofcontents

\newpage

\section{Introducción}
\label{sec:introduccion}
En este documento se recoge el desarrollo de la práctica final de Sistemas Distribuidos. Para esta práctica hemos desarrollado una aplicación distribuida que cuenta con 2 servidores desarrollados en lenguaje C, un código clinete desarrollado en python y un servicio web desarrollado igualmente en Python. A continuación describiremos el diseño e implementación de cada una de las partes del sistema. 
\section{Diseño original}
\label{sec:disenno}
La aplicación constará de dos partes diferenciadas: los clientes y el servidor.

\subsection{Cliente}
\label{subsec:cliente}
Aquí se habla del cliente 


\subsection{Servidor}
\label{subsec:servidor}
El servidor implementa los servicios necesarios para la coordinación de clientes. Para esto se apoya en una estructura de implementación propia especialmente diseñada para las particularidades de la práctica. 

\subsubsection{Implementación en el servidor}
\label{subsec:implementacion_servidor}
nuestra estructura vector que dobla de capacidad cuando se llena -> válido por su caracter de prototipo. 

\subsubsection{Concurrencia del Servidor}
\label{subsec::concurrencia_servidor}
Describir como aseguramos la concurrencia de clientes del lado del servidor => acceso a la estructura. 

\subsection{Comunicación}
\label{subsec:comunicacion}
Describir cómo hacemos la comunicación para que sea independiente del lenguaje 

\section{Servicio web}
\label{sec:web_service}
Describir aquí la implementación del servicio web y su integración.  

\section{Integración del servidor RPC}
\label{sec:rpc}
Describir aquí como hemos integrado el servidor y las modificaciones necesarias en el código 

\section{Compilación}
\label{sec:compilacion}
En esta sección nos centraremos en la forma de compilar los servidores, ya que son la única parte del código escrita en un lenguaje compilado. 

\section{Descripción de pruebas}
\label{sec:descripcion_de_pruebas}
Descripción de pruebas 

\section{Conclusiones}
\label{sec:conclusiones}
Este ejercicio combina casi todas las tecnologías que se nos han presentado durante el curso en un sistema completo que pretende aproximarse a una aplicación distribuída. Esto nos ha permitido afianzar los conocimientos adquiridos en la asignatura y desarrollar nuestras competencias para la programación de servicios distribuidos. 

Además, esta práctica nos presenta por primera vez la necesidad de integrar nuevos servicios basados en otra tecnología sobre un sistema ya funcional. Esta es una aptitud verdaderamente interesante y valiosa de cara a nuestro desarrollo como informáticos.
\end{document}
